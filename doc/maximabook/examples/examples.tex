%-*-EMaxima-*-

\subsection*{Establishing a Minimum for the Rayleigh Quotient}

We begin by defining the Rayleigh Quotient in general.  From basic
Regular Sturm-Liouville Eigenvalue principles, we know that the 
Rayleigh Quotient is defined as

\[\lambda =\frac{-p\phi \left. \frac{d\phi }{dx}\right| ^{b}_{a}+\int _{a}^{b}\, \left[p \left( \frac{d\phi }{dx}\right)^{2}-q\phi ^{2}\right] \, dx}{\int _{a}^{b}\, \phi ^{2}\sigma \, dx}\]


given the Sturm-Liouville differential equation

\[\frac{d}{dx}\left( p\left( x\right) \frac{d\phi }{dx}\right) +q\left( x\right) \phi +\lambda \sigma \left( x\right) \phi =0\]


where $a<x<b$.

\beginmaxima
RQ:(-p*('ev('ev(u(x)*'diff(u(x),x)),x=a)-'ev('ev(u(x)*diff(u(x),x)),x=b))+
'integrate(p*'diff(u(x),x)^2-q*u(x)^2,x,a,b))/'integrate(u(x)^2*sigma,x,a,b);
\maximatexoutput
{\small$$  {{\int_{a}^{b}{p\*\mathrm{\%DERIVATIVE}^{2}\left(\left(u\left(x
 \right),\linebreak[0]x,\linebreak[0]1\right)\right)-q\*u^{2}\left(
 \left(x\right)\right)\;dx}-p\*\left(\mathrm{EV}\left(\mathrm{EV}
 \left(u\left(x\right)\*\left({{d}\over{d\*x}}\*u\left(x\right)
 \right)\right),\linebreak[0]x=a\right)-\mathrm{EV}\left(\mathrm{EV}
 \left(u\left(x\right)\*\left({{d}\over{d\*x}}\*u\left(x\right)
 \right)\right),\linebreak[0]x=b\right)\right)}\over{\sigma\*\int_{a
 }^{b}{u^{2}\left(\left(x\right)\right)\;dx}}} $$
}
\endmaxima

Now we evaluate it.  This must be done in stages, otherwise the ev command
will not understand its arguements.

\beginmaxima
ev(RQ,p=1,q=0,sigma=1,u(x)=x-x^2,a=0,b=1);
ev(%,diff,integrate);
ev(%,ev);
\maximatexoutput
{\small$$   {{\mathrm{EV}\left(\mathrm{EV}\left(\left(x-x^{2}\right)\*\left(
 {{d}\over{d\*x}}\*\left(x-x^{2}\right)\right)\right),\linebreak[0]x=
 1\right)-\mathrm{EV}\left(\mathrm{EV}\left(\left(x-x^{2}\right)\*
 \left({{d}\over{d\*x}}\*\left(x-x^{2}\right)\right)\right)
 ,\linebreak[0]x=0\right)+\int_{0}^{1}{\mathrm{\%DERIVATIVE}^{2}
 \left(\left(x-x^{2},\linebreak[0]x,\linebreak[0]1\right)\right)\;dx}
 }\over{\int_{0}^{1}{\left(x-x^{2}\right)^{2}\;dx}}} $$
}
$$   30\*\left(\mathrm{EV}\left(\mathrm{EV}\left(\left(1-2\*x\right)
 \*\left(x-x^{2}\right)\right),\linebreak[0]x=1\right)-\mathrm{EV}
 \left(\mathrm{EV}\left(\left(1-2\*x\right)\*\left(x-x^{2}\right)
 \right),\linebreak[0]x=0\right)+{{1}\over{3}}\right) $$
$$   10 $$
\endmaxima

This can be checked by hand.  Seeing that it is correct, we now can use it to 
search for the minimum eigenvalue on a more difficult problem:

\beginmaxima
ev(RQ,p=1,q:-(x^2),sigma=1,u(x)=x-1,a=0,b=1)$
ev(%,diff,integrate)$
EV(%,EV,NUMER);
\maximatexoutput
$$   6.1 $$
\endmaxima

\beginmaxima
ev(RQ,p=1,q:-(x^2),sigma=1,u(x)=-2*x^2+2,a=0,b=1)$
ev(%,diff,integrate)$
ev(%,ev,NUMER);
\maximatexoutput
$$   2.642857142857143 $$
\endmaxima

\beginmaxima
ev(RQ,p=1,q:-(x^2),sigma=1,u(x)=x^3+x^2-2,a=0,b=1)$
ev(%,diff,integrate)$
ev(%,ev,NUMER);
\maximatexoutput
$$   2.776084010840108 $$
\endmaxima

The smallest eigenvalue must therefore be less than or equal to 2.642857...

\subsection*{Laplacian in Different Coordinate Systems}

This will probably go in the main documentation somewhere, but for now I'll
stick it here.

It is possible to express the Laplacian in different coordinate 
systems, provided you know how to define the coordinate system.
We will use Spherical Coordinates for our first example:

\beginmaxima
load(vect)$
scalefactors([[rho*cos(theta)*sin(phi),rho*sin(theta)*sin(phi),rho*cos(phi)],rho,theta,phi]);
depends(f,[rho,theta,phi]);
express(laplacian(f));
ev(%,diff)$
ratexpand(%);
\maximatexoutput
\p  Warning - you are redefining the MACSYMA function TRIGSIMP
Warning - you are redefining the MACSYMA function IMPROVE
Warning - you are redefining the MACSYMA function UPDATE \\
$$   \mathrm{DONE} $$
$$   \left[ f\left(\rho,\linebreak[0]\vartheta,\linebreak[0]\varphi
 \right) \right]  $$
$$   {{{{d}\over{d\*\rho}}\*\left({{d}\over{d\*\rho}}\*f\*\left| 
 \sin \varphi\right| \*\rho^{2}\right)+{{d}\over{d\*\vartheta}}\*{{{{
 d}\over{d\*\vartheta}}\*f\*\left| \sin \varphi\right| }\over{\sin ^{
 2}\varphi}}+{{d}\over{d\*\varphi}}\*\left({{d}\over{d\*\varphi}}\*f
 \*\left| \sin \varphi\right| \right)}\over{\left| \sin \varphi
 \right| \*\rho^{2}}} $$
$$   {{2\*\left({{d}\over{d\*\rho}}\*f\right)}\over{\rho}}+{{{{d
 }\over{d\*\varphi}}\*f\*\cos \varphi}\over{\sin \varphi\*\rho^{2}}}+
 {{{{d^{2}}\over{d\*\vartheta^{2}}}\*f}\over{\sin ^{2}\varphi\*\rho^{
 2}}}+{{{{d^{2}}\over{d\*\varphi^{2}}}\*f}\over{\rho^{2}}}+{{d^{2}
 }\over{d\*\rho^{2}}}\*f $$
\endmaxima
