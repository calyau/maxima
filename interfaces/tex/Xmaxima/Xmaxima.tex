% -*-mode: latex; fill-column: 75; tab-width: 8; coding: iso-latin-1-unix -*-
%
%	$Id: Xmaxima.tex,v 1.1 2002-09-24 00:32:47 mikeclarkson Exp $
%

\documentclass[12pt,makeidx,maxima]{report}

\newif\ifpdf\pdffalse
\ifx\pdfoutput\undefined\else\ifcase\pdfoutput
\else
  \pdftrue
  \input{pdfcolor}
  \pdfcompresslevel=9
  \pdfpagewidth=\paperwidth    % page width of PDF output
  \pdfpageheight=\paperheight  % page height of PDF output
  %
  % Pad the number with '0' to 3 digits wide so no page name is a prefix
  % of any other.
  %
\fi\fi
\ifpdf
\usepackage[pdftex]{graphicx}
\usepackage[pdftex]{epsfig}
\usepackage[pdftex]{color}
\usepackage[pdftex,colorlinks,pdfauthor={http://maxima.sourceforge.net/},pdftitle={Xmaxima User's Guide},urlcolor=blue,linkcolor=red,hyperindex]{hyperref}
\usepackage{nameref}
\fi

\usepackage{times}
\usepackage[T1]{fontenc}

\makeindex


\begin{document}
\title{Xmaxima User's Guide\\
	Version 5.9.0}

\author{Maxima Group,\\
        \ \\
	{\tt http://maxima.sourceforge.net}
        }

\date{Sept 2002}

\maketitle

\pagestyle{empty}
\clearpage
\vspace*{5in}
Printed \today.
\clearpage

\pagestyle{headings}
\pagenumbering{roman}
\tableofcontents
\clearpage

\clearpage
\chapter*{Preface}
\index{Preface}


\section*{Notational Conventions of this Manual}
\index{Notational Conventions of this Manual}

\pagenumbering{arabic}

\chapter{Installing Xmaxima}
\index{Installing Xmaxima}

\clearpage
\section{Installing Xmaxima under Windows}
\index{Installing Xmaxima under Windows}

\clearpage
\section{Installing Xmaxima under Unix}
\index{Installing Xmaxima under Unix}

\clearpage
\section{Environment Variables}
\index{Environment Variables}

\clearpage
\chapter{XMaxima User Interface}
\index{XMaxima User Interface}

\clearpage
\section{Menus}

\subsection{File Menu}

\subsection{Edit Menu}

\subsection{Options Menu}

\subsection{Maxima Menu}

\subsection{Help Menu}

\clearpage
\section{Console}

\clearpage
\section{Browser}

\clearpage
\chapter{Open Math}

\appendix
\clearpage
\chapter{Building Xmaxima for Windows}

\section{Installing MSYS}

\section{Wrapping the Tcl Code}

\section{Packaging with Inno}


\end{document}

